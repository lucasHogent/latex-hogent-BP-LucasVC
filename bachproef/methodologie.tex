%%=============================================================================
%% Methodologie
%%=============================================================================

\chapter{\IfLanguageName{dutch}{Methodologie}{Methodology}}%
\label{ch:methodologie}

%% TODO: In dit hoofstuk geef je een korte toelichting over hoe je te werk bent
%% gegaan. Verdeel je onderzoek in grote fasen, en licht in elke fase toe wat
%% de doelstelling was, welke deliverables daar uit gekomen zijn, en welke
%% onderzoeksmethoden je daarbij toegepast hebt. Verantwoord waarom je
%% op deze manier te werk gegaan bent.
%% 
%% Voorbeelden van zulke fasen zijn: literatuurstudie, opstellen van een
%% requirements-analyse, opstellen long-list (bij vergelijkende studie),
%% selectie van geschikte tools (bij vergelijkende studie, "short-list"),
%% opzetten testopstelling/PoC, uitvoeren testen en verzamelen
%% van resultaten, analyse van resultaten, ...
%%
%% !!!!! LET OP !!!!!
%%
%% Het is uitdrukkelijk NIET de bedoeling dat je het grootste deel van de corpus
%% van je bachelorproef in dit hoofstuk verwerkt! Dit hoofdstuk is eerder een
%% kort overzicht van je plan van aanpak.
%%
%% Maak voor elke fase (behalve het literatuuronderzoek) een NIEUW HOOFDSTUK aan
%% en geef het een gepaste titel.

Volgende hoofdstukken verlopen in sequentiële volgorde om dit onderzoek in de juiste richting te sturen.
De 


\section{Literatuurstudie}
Voor dit onderzoek heb ik eerst de vaktermen opgelijst en een onderverdeeld in groepen met gebruik van een mindmap.
Vervolgens zocht ik het internet af naar wetenschappelijke artikelen en filterde ik de nodige informatie.
\\
Het doel van de literatuurstudie is om een basis van bestaande kennis te verkrijgen door de belangrijkste concepten toe te lichten die van toepassing zijn op dit onderzoek. 
Dit houd in, het begrijpen van warehousing, WCS (Warehouse Control Systems), PLC (Programmable Logic Controllers), en de communicatieprotocollen die tussen deze systemen gebruikt worden.
\\
Hierdoor krijg je een overzicht van relevante theorieën, definities, en eerdere studies die inzicht geven in de werking en de gebruikte technologieën.
Hier werd voornamelijk gebruik gemaakt van wetenschappelijke artikelen, boeken en technische documenten.
\\
De literatuurstudie vormt de basis van het onderzoek en is cruciaal voor het begrijpen van de huidige stand van zaken.
Hierdoor kan je de kennis en requirements meenemen naar de vergelijkende studie.


\section{Analyse huidig messaging systeem}



Overzicht maken van het huidige systeem

Sterktes en zwaktes identificeren

Oplijsten van de niet functionele vereisten

Opzoeken van messaging technologien

Pro en con definieren per technologie

Opzoeken van test methodes 

Testen en monitoring opzetten per gekozen technologie

Evalueren
