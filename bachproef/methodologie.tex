%%=============================================================================
%% Methodologie
%%=============================================================================

\chapter{\IfLanguageName{dutch}{Methodologie}{Methodology}}%
\label{ch:methodologie}

%% TODO: In dit hoofstuk geef je een korte toelichting over hoe je te werk bent
%% gegaan. Verdeel je onderzoek in grote fasen, en licht in elke fase toe wat
%% de doelstelling was, welke deliverables daar uit gekomen zijn, en welke
%% onderzoeksmethoden je daarbij toegepast hebt. Verantwoord waarom je
%% op deze manier te werk gegaan bent.
%% 
%% Voorbeelden van zulke fasen zijn: literatuurstudie, opstellen van een
%% requirements-analyse, opstellen long-list (bij vergelijkende studie),
%% selectie van geschikte tools (bij vergelijkende studie, "short-list"),
%% opzetten testopstelling/PoC, uitvoeren testen en verzamelen
%% van resultaten, analyse van resultaten, ...
%%
%% !!!!! LET OP !!!!!
%%
%% Het is uitdrukkelijk NIET de bedoeling dat je het grootste deel van de corpus
%% van je bachelorproef in dit hoofstuk verwerkt! Dit hoofdstuk is eerder een
%% kort overzicht van je plan van aanpak.
%%
%% Maak voor elke fase (behalve het literatuuronderzoek) een NIEUW HOOFDSTUK aan
%% en geef het een gepaste titel.

Volgende hoofdstukken verlopen in sequentiële volgorde om dit onderzoek in de juiste richting te sturen.
De literatuurstudie geeft een basis om verdere vakterminologie en werking van technologieën te kunnen begrijpen.
\\
De setup van het huidige systeem wordt besproken om de requirements te kunnen begrijpen.
Het bekomen van een shortlist wordt toegelicht waarbij voor elke technologie de voor- en nadelen wordt beschreven.


\section{Phase 1: Literatuurstudie}
Voor dit onderzoek heb ik eerst de vaktermen opgelijst en onderverdeeld in groepen met behulp van een mindmap.
Vervolgens werd er gebruik gemaakt van het internet om wetenschappelijke artikelen op te zoeken en filterde ik de nodige informatie.
\\
Het doel van de literatuurstudie is om een basis van bestaande kennis te verkrijgen door de belangrijkste concepten toe te lichten die van toepassing zijn op dit onderzoek. 
Dit houd in, het begrijpen van warehousing, WCS (Warehouse Control Systems), PLC (Programmable Logic Controllers), en de communicatieprotocollen die tussen deze systemen gebruikt worden.
\\
Hierdoor krijg je een overzicht van relevante theorieën, definities, en eerdere studies die inzicht geven in de werking en de gebruikte technologieën.
Hier werd voornamelijk gebruik gemaakt van wetenschappelijke artikelen, boeken en technische documenten.
\\
De literatuurstudie vormt de basis van het onderzoek en is cruciaal voor het begrijpen van de huidige stand van zaken.
Hierdoor kan de kennis en requirements meegenomen worden doorheen de methodologie.


\section{Phase 2: Requirements analyse van huidige setup}
Dit hoofdstuk gaat dieper in op de huidige setup en werking tussen het WCS, messaging software en de PLC's.
Documenten van het bedrijf en interviews met vakexperts vormen de basis voor dit overzicht.
\\
Het doel is om na het lezen van dit hoofdstuk inzicht te krijgen in de requirements van het huidige systeem, 
zodat deze kunnen dienen als basis voor de keuze van een surrogaat van het huidige messaging systeem.
\\

\subsection{PLC gebruik binnen TVH}
Er zijn 7 verschillende PLC's in TVH Waregem die instaan voor verschillende zones van de conveyor.
Deze werden aangeleverd door Vanderlanden in het jaar 2013 en worden beheerd door het automatisatie team.
De communicatie tussen een PLC en het WCS is gebaseerd op het TCP/IP protocol en is verbonden via het intern netwerk.
Er is een tussenlaag tussen de PLC en het netwerk, RFC1006 van het merk Siemens waarin configuratie kan worden gedaan door het automatisatie team.
Dit stelt de collega's in staat om bepaalde logica te implementeren of netwerk aanpassingen door te voeren.
De snelheid van communicatie is essentieel, daarom moet het netwerk snel genoeg zijn zodat berichten aan een snel tempo verstuurd kunnen worden.

De PLC's maken gebruik van een TCP/IP-socketverbinding en functioneren als client ten opzichte van het WCS, dat de rol van server vervult. 
Dit betekent dat de PLC de verbinding initieert en persisteert met de server die verantwoordelijk is voor de communicatie.
Een PLC is verantwoordelijk voor een specifieke zone van de conveyor en is opgebouwd uit drie kanalen die elk via een toegewezen poortnummer met de server communiceren. 
Meerdere kanalen zijn nodig om de communicatiesnelheid te bevorderen en omdat elk kanaal zijn eigen type informatie verwerkt.

\begin{table}[!h]
  \centering
  \begin{tabular}{lcr}
    \toprule
    \textbf{Kanaal} & \textbf{Beschrijving} & \textbf{Type}                \\
    \midrule
    1                & Route informatie over transportbak          & Snel           \\
    2                & Informatie van PLC                          & Niet kritisch  \\
    3                & Overige informatie over transportbak        & Snel           \\
    \bottomrule
  \end{tabular}
  \caption[Channel assignment]{\label{tab:channel-assignment}Beschrijving van kanalen}
\end{table}

\subsubsection{PLC berichten}
Berichten bestaan uit een frame opgedeeld in velden en hebben een specifieke lengte.
De inhoud van een bericht is gebaseerd op het hexadecimale stelsel en wordt in detail toegelicht in de onderstaande tabel.
Bepaalde controles worden uitgevoerd om de validiteit van een bericht af te toetsen. 
\\\\
Er worden ongeveer 80 berichten per minuut verstuurd per PLC, per kanaal.
Het totale aantal verstuurde berichten per minuut komt uit op ongeveer 1680/min.

\begin{table}[h!]
\centering 
\begin{tabular}{|c|c|c|c|}
  \hline
  \textbf{Veld} & \textbf{Inhoud} & \textbf{Data type} & \textbf{Lengte} \\
  % \hline
  % Dummy & Enkel PLC naar WCS
  \hline 
  Header & <STX> & Binair & 1 byte \\
  \hline 
  Lengte in bytes & 001D(HEX) & Binair & 2 bytes \\
  \hline 
  Seq. nummer &  [0-9] & ASCII & 1 byte  \\
  \hline 
  Inhoud & <...> & Binair & 27 bytes \\
  \hline 
  Terminator & <ETX> & Binair & 1 byte \\
  \hline
\end{tabular}
\caption[Message content]{\label{tab:message-content}Inhoud bericht}
\end{table}

Voorbeeld van een bericht dat van PLC naar WCS wordt verstuurd: 
\begin{listing}[h!]
\begin{minted}{python}
  02 00 1d 20 30 36 20 20 00 00 20 20 30 37 20 20 30 20 20 20 20 20 20 20 20 20 20 20 20 20 20 03
\end{minted}
\caption[Voorbeeld PLC bericht]{Voorbeeld van een PLC bericht}
\end{listing}

\subsection{Middleware tussen PLC en WCS}
De PLC kan alleen maar een TCP/IP socket verbinding initiëren met een server.
Omdat SonicMQ als middleware hierdoor geen verbinding kan maken zijn er listeners gemaakt in Java door TVH.
Deze listeners fungeren als server en zijn specifiek opgesteld om een TCP/IP socket verbinding mogelijk te maken per PLC kanaal.
De Java listeners sturen de PLC-berichten vervolgens door naar SonicMQ of ontvangen berichten van SonicMQ, 
die ze via een socket naar de PLC doorsturen.
Aan de kant van het WCS zijn er meer mogelijkheden om verbinding te kunnen maken met een server.
\\\\
SonicMQ bied geen support meer en is niet populair waardoor er ook geen community is.

\subsection{WCS communicatie} 
Op het ERP-systeem van TVH draaien acht verschillende batches die verantwoordelijk zijn voor de aansturing van de PLC. 
Elke batch-instantie communiceert met specifieke PLC-kanalen en bevat daarvoor specifieke logica, geschreven in OpenEdge Progress 4GL.
Deze batches zijn verbonden via een specifieke poort met een \textbf{Progress JMS Adapter} op de communicatie server omdat ze de ``Broker connect'' methode gebruiken.
Hiermee kunnen de batches de berichten consumeren en versturen van de SonicMQ server.
Daarnaast is het ook mogelijk om de Client connect methode te gebruiken, wat gemakkelijker is qua integratie.

\subsubsection{Progress OpenEdge JMS Adapter}
De adapter stelt OpenEdge-applicaties in staat om berichten te verzenden en te ontvangen van JMS (Java Messaging System)
message brokers zoals Apache ActiveMQ, IBM MQ, en TIBCO EMS. 
Dit betekent dat OpenEdge-applicaties kunnen integreren met andere systemen die JMS ondersteunen, 
zonder dat er directe afhankelijkheden nodig zijn.
Hierdoor zijn we gelimiteerd in de keuze van message brokers tot brokers die JMS ondersteunen.
\\
\\
Omdat JMS asynchrone communicatie ondersteunt, kunnen OpenEdge-applicaties berichten verzenden zonder te wachten op een onmiddellijke respons.
Dit is nuttig voor werkprocessen die niet direct afhangen van real-time gegevens, zoals orderverwerking, 
melding van gebeurtenissen, en batchverwerkingen.
\\\\
Er is ondersteuning van Publish/Subscribe en Point-to-Point Modellen.
Bij een point-to-point communiceert één afzender met één ontvanger, bijvoorbeeld een opdracht naar een specifieke applicatie.
\\\\
Bij het publish/subscribe model kunnen meerdere ontvangers zich abonneren op bepaalde berichten (topics) die door een applicatie gepubliceerd worden.

Door JMS te gebruiken met message brokers die schaling en betrouwbaarheid ondersteunen, kan de OpenEdge-omgeving hoge verwerkingsvolumes en failover-functionaliteit realiseren. 
Dit is belangrijk voor bedrijfsomgevingen waar de continuïteit en betrouwbaarheid van berichtenstromen essentieel zijn.

\subsubsection{Integratie met Verschillende Enterprise Systemen}
Met de JMS Adapter kunnen OpenEdge-applicaties communiceren met andere systemen zoals ERP's, CRM’s, wat nuttig is in bedrijfsprocessen waarbij 
gegevens moeten worden gedeeld tussen verschillende systemen.

\subsubsection{Configuratie en Beheer}
De Progress OpenEdge JMS Adapter biedt configuratiemogelijkheden waarmee beheerders de communicatie kunnen aanpassen aan de vereisten van hun omgeving, 
zoals het instellen van queue-namen, topics, verbindingsparameters, en het beheren van uitzonderingen.

\subsubsection{WCS berichten} 
Berichten komen binnen van de PLC via de communicatie server. Ieder bericht wordt getransformeerd naar variabelen die dan verder gebruikt worden in de code.
Deze berichten bevatten informatie over transportbakken en zijn nodig om deze te kunnen traceren via de ERP.
Specifieke logica is nodig om bakken tot hun bestemming te krijgen, of om fout afhandeling te voorzien.
Volgende voorbeelden doen zich voor:
\begin{enumerate}
\item Routeren naar een hospitaal punt door: 
\begin{enumerate}
  \item Gewichtsfout
  \item Hoogtefout
  \item Onbekende bestemming
\end{enumerate}
\item Bestemming wordt gevraagd door de PLC
\item Bestemming wordt doorgegeven aan de PLC 
\item Specifieke logica moet uitgevoerd worden bij het passeren van een bepaald punt
\item \dots
\end{enumerate}

\subsection{Monitoring}
Het bestaande systeem wordt gemonitord met behulp van software zoals Prometheus, Grafana en Elastic.
Hierdoor kunnen systeemfouten snel opgemerkt worden en berichten verstuurd worden als bepaalde waardes overschreden worden.

\subsection{Samenvatting requirements messaging systeem}
De belangrijkste requirements voor de huidige setup zijn als volgt:

\subsubsection{Integratie met het WCS systeem}
Het messaging systeem moet kunnen integreren met het huidige WCS-systeem dat gebruik maakt van Progress 4GL versie 11.7. 
Hiervoor moet de broker het \textbf{JMS protocol ondersteunen} en geïnstalleerd kunnen worden op een \textbf{Unix platform}.
Daarnaast moet de middleware ook gebruikt kunnen worden door \textbf{monitoring software}.

\subsubsection{Performantie}
Om de real-time eisen van het WCS en de PLC’s te ondersteunen, moet het messaging systeem \textbf{lage latentie} bieden. 
Dit betekent dat berichten zonder merkbare vertraging moeten worden verstuurd en ontvangen, 
zodat de snelheid van de conveyor niet wordt beperkt door de communicatiesnelheid.
Om vertraging te voorkomen moet de verwerking van de berichten \textbf{asynchroon} gebeuren.

\subsubsection{Betrouwbare Berichtenoverdracht}
Het systeem moet in staat zijn berichten \textbf{consistent en zonder verlies} over te brengen. 
Dit is essentieel om de traceerbaarheid van transportbakken te garanderen en fouten in de logistieke processen te voorkomen.
Hiervoor moet de broker het \textbf{AMQP protocol} ondersteunen.

\subsubsection{Support en Community}
Het is belangrijk dat de gekozen message broker support biedt en dat er een grote community aanwezig is 
waarbij je terecht kunt voor advies, probleemoplossing en best practices.

\subsubsection{Kosten}
Support gaat vaak samen met kosten en is ook belangrijk om mee te nemen in de keuze van een messaging systeem.
Sommige systemen hangen vast aan een kostenmodel, andere zijn open-source en brengen geen kosten met zich mee.

\section{Phase 3: Long List}
In dit hoofdstuk vertrekken we van een lijst met alle message brokers die beschikbaar zijn op de markt.
Deze worden gefilterd door bepaalde criteria toe te passen, waarna de overgebleven kandidaten getest worden.
Op deze manier worden alle mogelijke opties geargumenteerd.

\subsection{Beschikbare messaging software}
Onderstaande lijst geeft een overzicht van de meest populaire messaging brokers beschikbaar op de markt.
Met behulp van argumenten worden relevante systemen gefilterd die voldoen aan volgende criteria om te kunnen integreren met het huidige WCS:
\begin{itemize}
\item JVM: JVM ondersteuning beschikbaar
\item Unix: Installatie mogelijk op Unix OS 
\item Protocol: Minstens het AMQP protocol
\item Kostenmodel: open source  
\item On premise: Moet lokaal kunnen geïnstalleerd worden
\item Actieve Community en support
\end{itemize}

\begin{table}[h!]
\footnotesize
\centering
\begin{tabular}{|l|c|c|c|c|c|}
\hline
Message Broker & JVM & Unix & Protocollen & Kostenmodel & On premise \\
\hline
Apache Kafka & Ja & Ja & Kafka, MQTT, REST & Open-source & Ja \\
\hline
RabbitMQ & Ja & Ja & AMQP, MQTT, STOMP & Open-source & Ja \\
\hline
ActiveMQ & Ja & Ja & AMQP, MQTT, STOMP, OpenWire & Open-source & Ja \\ 
\hline
Artemis & Ja & Ja & AMQP, MQTT, STOMP & Open-source & Ja \\
\hline
MQTT.js & Ja & Ja & MQTT & Open-source & Ja \\
\hline
IBM MQ & Ja & Ja & MQ, MQTT & Abonnement & Ja \\
\hline
Redis & Ja & Ja & Stream, Pub/Sub & Open-source x & Ja \\
\hline
NSQ & Ja & Ja & Stream, Pub/Sub & Open-source & Ja \\
\hline
Apache Pulsar & Ja & Ja & Stream, Pub/Sub & Open-source & Ja \\
\hline
NATS & Ja & Ja & NATS & Open-source & Nee \\
\hline
ZeroMQ & Ja & Ja & ØMQ (Socket) & Open-source & Ja \\ 
\hline
Amazon SQS & Nee & Nee & AWS protocol & Pay-per-use & Nee \\
\hline
Google Cloud Pub/Sub & Nee & Nee & Cloud Pub/Sub & Pay-per-use & Nee \\
\hline
Azure Service Bus & Nee & Nee & AMQP, MQTT, HTTP & Pay-per-use & Nee \\
\hline
\end{tabular}
\caption{\label{tab:message_brokers}Longlist message brokers}
\end{table}

\subsubsection{Apache Kafka}
Apache Kafka beschikt over een grote en actieve community waarvan diverse bedrijven betaalde ondersteuning aanbieden.
Kafka wisselt data uit via streams en ondersteunt geen AMQP-protocol. 
Communicatie met Kafka is mogelijk via een Kafka-adapter in Progress OpenEdge 12.8.
Aangezien het WCS draait op Progress OpenEdge 11.7, is Kafka niet compatibel met de huidige setup.
Hierdoor voldoet deze broker niet aan de gestelde requirements en kan deze niet als optie worden gekozen.
Ondanks deze technologie momenteel niet kan gebruikt worden, kan het wel een kandidaat zijn in de toekomst.

\subsubsection{RabbitMQ}
RabbitMQ is een open source message broker die geïnstalleerd kan worden op diverse Unix-distributies en een breed scala aan mogelijkheden biedt.
Deze broker ondersteunt functies zoals routering, filtering, streaming, en meer.
Routering naar queues is mogelijk via verschillende protocollen, waaronder AMQP, wat RabbitMQ zeer veelzijdig maakt.
Daarnaast beschikt RabbitMQ over uitgebreide documentatie, actieve ontwikkeling, een grote community, en de mogelijkheid tot professionele ondersteuning.
Omdat RabbitMQ JVM-ondersteuning biedt, is het geschikt voor onze shortlist en voldoet het aan de gestelde eisen.

\subsubsection{ActiveMQ Classic}
ActiveMQ Classic is een populaire JVM-gebaseerde message broker met een grote en actieve community.
Het kan eenvoudig worden geïnstalleerd in zowel lokale als cloudomgevingen en functioneert uitstekend op moderne Unix-systemen.
De broker ondersteunt meerdere protocollen, waaronder AMQP, naast OpenWire, STOMP, en MQTT, wat het veelzijdig maakt in uiteenlopende toepassingen.
Daarnaast is ActiveMQ Classic gratis en vereist het geen licentiekosten, wat het bijzonder aantrekkelijk maakt voor bedrijven met een beperkt budget.
Hoewel ActiveMQ Classic een stabiele en bewezen oplossing is, verschuift het momentum van de community langzaam richting ActiveMQ Artemis, 
waardoor toekomstige innovaties minder frequent kunnen zijn.
Desondanks blijft het een betrouwbare keuze voor organisaties die op zoek zijn naar een kostenbesparende en robuuste message broker.

\subsubsection{ActiveMQ Artemis}
ActiveMQ Artemis is een open source JVM-gebaseerde broker, vergelijkbaar met ActiveMQ Classic, maar met een andere interne architectuur.
Artemis gebruikt een eigen opslagmechanisme dat sneller is dan KahaDB, de opslaglaag die door ActiveMQ Classic wordt gebruikt.
Berichten worden sequentieel opgeslagen, waardoor er geen aparte index nodig is.
Dit vermindert de kans op prestatieverlies, zelfs bij het verwerken van grote hoeveelheden berichten.
De broker ondersteunt verschillende protocollen, waaronder AMQP, waardoor Artemis geschikt is voor diverse toepassingen.
Zowel community support als commerciële ondersteuning via verschillende bedrijven is beschikbaar, 
wat flexibiliteit biedt in onderhoud en beheer.
Dit product komt ook in aanmerking met de vereisten en nemen we mee in de shortlist

\subsubsection{MQTT.js}
MQTT.js is een open source JavaScript-bibliotheek voor het implementeren van het MQTT-protocol in zowel Node.js- als browseromgevingen.
Hierdoor is deze technologie niet gekozen worden voor de huidige setup.

\subsubsection{IBM MQ}
IBM MQ is een JVM-gebaseerde message broker en is geschikt voor Unix OS-omgevingen en kan draaien op zowel on-premises als cloud omgevingen.
IBM MQ ondersteunt meerdere protocollen, waaronder AMQP, wat het flexibel maakt in hybride omgevingen waar interoperabiliteit met verschillende systemen vereist is.
De broker biedt enterprise-grade functionaliteit met ondersteuning voor transacties, garanties voor berichtlevering, en beveiliging, wat het ideaal maakt voor veeleisende toepassingen.
Wat betreft het kostenmodel biedt IBM MQ licentiemodellen op basis van vier categorieën, 
met verschillende opties voor zowel lokale implementaties als cloud gebaseerde oplossingen.
De kosten kunnen relatief hoog zijn, vooral in grotere omgevingen, maar bieden wel uitgebreide functionaliteiten en garanties.
IBM MQ heeft een gesloten community en biedt ondersteuning via IBM zelf. 
Dit maakt het een goede keuze voor organisaties die behoefte hebben aan robuuste, goed ondersteunde messaging-oplossingen.
Voor het WCS kan deze technologie als overmatig gezien worden omdat de vele functionaliteiten niet toegepast worden.

\subsubsection{Redis}
Redis is een in-memory data store dat kan gebruikt worden als message broker dat gebruik maakt van streams of pub/sub via API. 
Het ondersteunt JVM-integratie via verschillende client bibliotheken en is compatibel met Unix OS-omgevingen, 
wat het geschikt maakt voor zowel lokale als cloud implementaties.
Hoewel Redis oorspronkelijk geen specifiek messaging-protocol zoals AMQP ondersteunt, 
kan het berichtenuitwisseling realiseren via pub/sub-mechanismen. 
Wat betreft het kostenmodel, Redis is open source en gratis te gebruiken, wat het aantrekkelijk maakt voor organisaties met beperkte budgetten. 
Redis biedt ook cloud-gebaseerde services via aanbieders zoals Redis Labs, met premium opties voor extra functionaliteit en ondersteuning.
De community rond Redis is groot en actief, met veel beschikbare documentatie, tutorials en third-party tools. 
Voor bedrijven die commerciële ondersteuning nodig hebben, biedt Redis Labs betaalde ondersteuning en aanvullende services, waaronder geavanceerde beveiliging en schaalbaarheid.

\subsubsection{NSQ}
NSQ is een open source, gedistribueerde message broker die ontworpen is voor hoge beschikbaarheid en schaalbaarheid. 
Het is geschikt voor JVM-integratie via verschillende client bibliotheken en werkt op Unix OS-omgevingen, 
wat het geschikt maakt voor zowel lokale als cloud gebaseerde implementaties.
NSQ is ontworpen om berichtenverkeer op grote schaal te verwerken en biedt geen AMQP-protocol, enkel ondersteuning voor pub/sub-communicatie. 
Het kostenmodel van NSQ is open source en volledig gratis te gebruiken. 
Er zijn geen licentiekosten, en de schaalbaarheid van het systeem maakt het geschikt voor zowel kleine als grote omgevingen.
NSQ heeft een actieve community die regelmatig updates en ondersteuning biedt via GitHub en andere open source platforms. 
Voor bedrijven die behoefte hebben aan commerciële ondersteuning, zijn er diverse consultants en serviceproviders die ondersteuning bieden.

\subsubsection{Apache Pulsar}
Apache Pulsar is een project gestart door Yahoo en open source sinds 2016.
Dit is een gedistribueerde messaging systeem dat ondersteuning biedt voor zowel JVM-omgevingen als Unix OS-systemen.
Het is geschikt voor zowel cloud- als lokale omgevingen. 
Pulsar ondersteunt publish-subscribe en queue-based messaging.
AMQP kan gebruikt worden via een connector maar brengt hierdoor extra complexiteit met zich mee.
Deze software wordt gebruikt voor meer complexere use-cases en heeft een ingewikkelde setup.
De community rondom Apache Pulsar is actief, met regelmatige updates en ondersteuning via de Apache Software Foundation. 
Commerciële ondersteuning is beschikbaar via bedrijven die Pulsar als platform aanbieden.
\\
Bron: https://www.baeldung.com/apache-pulsar

\subsubsection{NATS}

\subsubsection{ZeroMQ}

\subsubsection{Amazon SQS}

\subsubsection{Google Cloud Pub/Sub}

\subsubsection{Azure Service Bus}

\newpage
\subsection{Compatibele messaging software}
Volgende tabel is een gefilterde lijst van de beschikbare messaging brokers.
Deze selectie wordt getest door deze te implementeren in een gesimuleerde omgeving.
Hierdoor kan het installatie process gedocumenteerd worden en de prestaties gemeten worden om tot de shortlist te komen.

\begin{table}[h!]
\footnotesize
\centering
\begin{tabular}{|l|c|c|c|c|}
\hline
Message Broker & Prestatie & Ondersteuning & Kostenmodel \\
\hline
RabbitMQ & Ja & Ja & AMQP, MQTT, STOMP \\
\hline
ActiveMQ & Ja & Ja & AMQP, MQTT, STOMP, OpenWire \\ 
\hline
Artemis & Ja & Ja & AMQP, MQTT, STOMP  \\
\hline
Azure Service Bus & Ja & Ja & MQ, MQTT, JMS  \\
\hline
\end{tabular}
\caption{\label{tab:message_brokers_filtered}Longlist compatibele message brokers}
\end{table}

\section{Phase 4: Testen}

\subsection{Systeem Hardware Informatie}
\subsubsection*{Geheugen}
\begin{itemize}
    \item \textbf{Omschrijving:} System memory
    \item \textbf{Fysieke ID:} 0
    \item \textbf{Grootte:} 8GiB
\end{itemize}

\subsubsection*{CPU}
\begin{itemize}
    \item \textbf{Product:} Intel(R) Xeon(R) Gold 6154 CPU @ 3.00GHz
    \item \textbf{Fabrikant:} Intel Corp. 
    \item \textbf{Versie:} 6.85.0
    \item \textbf{Breedte:} 64 bits
\end{itemize}

\subsection{Test ActiveMQ}

\section{Phase 5: ShortList}

Artemis, ActiveMQ, RabbitMQ
\begin{table}[h!]
  \footnotesize
  \centering
  \begin{tabular}{|l|c|c|c|c|}
  \hline
  & Artemis & ActiveMQ & RabbitMQ \\
  \hline
  Ondersteuning &  &  &  \\
  \hline
  Performantie &  &  &  \\
  \hline
  Integratie &  &  &  \\ 
  \hline
  Kosten &   &  &  \\
  \hline
  Score &   &  &  \\
  \hline
    
  \end{tabular}
  \caption{\label{tab:message_brokers_short}Shortlist message brokers}
  \end{table}








