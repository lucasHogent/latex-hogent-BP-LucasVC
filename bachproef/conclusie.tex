%%=============================================================================
%% Conclusie
%%=============================================================================

\chapter{Conclusie}%
\label{ch:conclusie}

% TODO: Trek een duidelijke conclusie, in de vorm van een antwoord op de
% onderzoeksvra(a)g(en). Wat was jouw bijdrage aan het onderzoeksdomein en
% hoe biedt dit meerwaarde aan het vakgebied/doelgroep? 
% Reflecteer kritisch over het resultaat. In Engelse teksten wordt deze sectie
% ``Discussion'' genoemd. Had je deze uitkomst verwacht? Zijn er zaken die nog
% niet duidelijk zijn?
% Heeft het onderzoek geleid tot nieuwe vragen die uitnodigen tot verder 
%onderzoek?

De drie kandidaten die aan de testen werden onderworpen, voldeden allemaal aan de vooraf gestelde eisen.
Tussen deze kandidaten zijn duidelijke verschillen te merken waardoor één kandidaat als optimale message broker kan gekozen worden.

\section{Apache ActiveMQ}
Apache ActiveMQ Classic is een snelle en kortetermijnoplossing, 
omdat deze met minimale aanpassingen eenvoudig te integreren is in het huidige systeem.

\section{RabbitMQ}
RabbitMQ is een modern messaging-systeem dat zowel toekomstbestendig als zeer krachtig is. 
In tegenstelling tot ActiveMQ en Artemis, biedt RabbitMQ geen ingebouwde filtering, wat het minder geschikt kan maken voor de huidige opstelling.
RabbitMQ heeft een grote community en wordt het actief ontwikkeld. Het kan een robuuste oplossing zijn, 
mits de huidige opstelling wordt aangepast.

\section{Apache Artemis}
Apache Artemis komt naar voor als beste van de drie kandidaten:
\begin{itemize}
    \item \textbf{Performantie:} Uit de performantie resultaten blijkt dat Artemis duidelijk sneller is dan ActiveMQ
    \item \textbf{Functionaliteit:} Filtering kan toegepast worden in tegenstelling tot RabbitMQ.
    \item \textbf{Toekomstbestendig} Als modernere versie van ActiveMQ is Artemis beter geschikt voor langdurig gebruik.
    \item \textbf{Implementatiegemak:} Vereist minimale aanpassingen om te integreren in tegenstelling tot RabbitMQ.
    \item \textbf{Schaalbaarheid:} Het systeem ondersteunt meerdere instanties, wat meer schaalbaarheid biedt dan ActiveMQ.
\end{itemize}  

\section{Conclusie}
Hoewel alle kandidaten voldoen aan de gestelde eisen, biedt Apache Artemis de beste balans tussen prestaties, 
functionaliteit, schaalbaarheid en implementatiegemak. 
Dit maakt het de meest geschikte keuze voor zowel de huidige als toekomstige behoeften.



