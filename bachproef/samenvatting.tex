%%=============================================================================
%% Samenvatting
%%=============================================================================

% TODO: De "abstract" of samenvatting is een kernachtige (~ 1 blz. voor een
% thesis) synthese van het document.
%
% Een goede abstract biedt een kernachtig antwoord op volgende vragen:
%
% 1. Waarover gaat de bachelorproef?
% 2. Waarom heb je er over geschreven?
% 3. Hoe heb je het onderzoek uitgevoerd?
% 4. Wat waren de resultaten? Wat blijkt uit je onderzoek?
% 5. Wat betekenen je resultaten? Wat is de relevantie voor het werkveld?
%
% Daarom bestaat een abstract uit volgende componenten:
%
% - inleiding + kaderen thema
% - probleemstelling
% - (centrale) onderzoeksvraag
% - onderzoeksdoelstelling
% - methodologie
% - resultaten (beperk tot de belangrijkste, relevant voor de onderzoeksvraag)
% - conclusies, aanbevelingen, beperkingen
%
% LET OP! Een samenvatting is GEEN voorwoord!

%%---------- Nederlandse samenvatting -----------------------------------------
%
% TODO: Als je je bachelorproef in het Engels schrijft, moet je eerst een
% Nederlandse samenvatting invoegen. Haal daarvoor onderstaande code uit
% commentaar.
% Wie zijn bachelorproef in het Nederlands schrijft, kan dit negeren, de inhoud
% wordt niet in het document ingevoegd.

\IfLanguageName{english}{%
\selectlanguage{dutch}
\chapter*{Samenvatting}
% \lipsum[1-4]



\selectlanguage{english}
}{}

%%---------- Samenvatting -----------------------------------------------------
% De samenvatting in de hoofdtaal van het document

\chapter*{\IfLanguageName{dutch}{Samenvatting}{Abstract}}

% \lipsum[1-4]

% Intro (use case doelpubliek en doelstelling) 
% Kaderen thema
Dit onderzoek situeert zich binnen de IT afdeling van TVH (Thermote \& Vanhalst). 
TVH is een bedrijf met als hoofdactiviteit het verkopen van onderdelen in de logistieke sector.
Het bedrijf bestaat uit meerdere automatische magazijnen die instaan voor de opslag van goederen. 
Een automatisch magazijn bestaat uit meerdere IT componenten genaamd services, die elk hun specifieke verantwoordelijkheden hebben. 
Om de gehele werking te kunnen garanderen moeten deze componenten met elkaar kunnen communiceren. 
\newline

% Probleemstelling
De huidige Unix-server die instaat voor de communicatie, maakt gebruik van \mbox{CentOS} als besturingssysteem,
wordt niet ondersteund en is verouderd.
Hierdoor is de server onderhevig aan veiligheidsrisico's en moet die vervangen worden door een server met een Redhat OS, 
die wel officiële ondersteuning aanbiedt.
De gebruikte messaging software, SonicMQ, is eveneens verouderd en kan niet geïnstalleerd worden op de nieuwe server. 
\newline

Omdat vervanging noodzakelijk is, is de vraag in dit onderzoek: 
Welke modern messaging systeem kan het huidige SonicMQ-systeem tussen de PLC en het WMS vervangen, 
met aandacht voor compatibiliteit, efficiëntie, betrouwbaarheid, veiligheid en kost?
Deze vraag vormt de leidraad voor het onderzoek met als doel 
% doel
de SonicMQ messaging-systeem te vervangen door een moderner, future-proof systeem.
Naast de integratie van de nieuwe software, moet deze ook voldoen aan de vereisten
om de continuïteit van het automatisch magazijn te waarborgen. 
Daarom is dit onderzoek nodig om de meest geschikte technologieën te identificeren met als doel de optimale te adviseren.
\newline

% Opbouw paper
Om een duidelijk overzicht te bieden zal deze paper eerst dieper ingaan op de kernbegrippen 
om de context te verduidelijken waarin de bachelorproef zal worden uitgevoerd.
\newline

% Methode
Om een doordachte keuze te kunnen maken, worden verschillende messaging-systemen onderzocht en vergeleken.
Hiervoor is een shortlist opgesteld waarin elke technologie wordt afgetoetst.
De geselecteerde technologieën worden aan verschillende testen onderworpen in een virtuele omgeving die de communicatie van het automatische magazijn simuleert. 
Op deze manier kan een weloverwogen keuze worden gemaakt die voldoet aan de non-functional requirements.