%%=============================================================================
%% Inleiding
%%=============================================================================

\chapter{\IfLanguageName{dutch}{Inleiding}{Introduction}}%
\label{ch:inleiding}

% De inleiding moet de lezer net genoeg informatie verschaffen om het onderwerp te begrijpen en in te zien waarom 
% de onderzoeksvraag de moeite waard is om te onderzoeken. 
% In de inleiding ga je literatuurverwijzingen beperken, zodat de tekst vlot leesbaar blijft. 
% Je kan de inleiding verder onderverdelen in secties als dit de tekst verduidelijkt. 
% Zaken die aan bod kunnen komen in de inleiding~\autocite{Pollefliet2011}:

% \begin{itemize}
%   \item Context
%   \item Het onderwerp
%   \item Verantwoording van het onderwerp, methodologie
%   \item Probleemstelling
%   \item Onderzoeksdoelstelling
%   \item Onderzoeksvraag
%   \item \ldots
% \end{itemize}

\section{\IfLanguageName{dutch}{Context}{Context}}%
\label{sec:context}
% Context
TVH is een retail bedrijf dat zich focust op de verkoop van onderdelen binnen de logistieke sector. 
Bedrijven zoals TVH hebben verschillende domeinen en subdomeinen met specifieke doeleinden binnen de organisatie. 
Deze bachelorproef speelt zich af binnen het domein "warehousing", dat instaat voor het beheer van goederen.
Warehousing in TVH bevat verschillende subcomponenten waaronder het WMS (Warehouse Management System) en WCS (Warehouse Control System).
Goederen kunnen naar een bepaalde bestemming in het gebouw getransporteerd worden via een transportband, genaamd conveyor.
Hiervoor is er communicatie nodig in beide richtingen tussen het WCS (Warehouse Control System) en de PLC's (Programmable Logic Controllers).
De PLC's voor het transportsysteem worden beheerd door techniekers van het automatisatie team.
Het WCS systeem, geschreven in Progress 4GL-code en de server die verantwoordelijk is voor de communicatie worden beheerd door IT.
\newline
\newpage

\section{\IfLanguageName{dutch}{Het onderwerp}{Het onderwerp}}%
\label{sec:Het onderwerp}
% Afbakening
% Server
Omdat het transportsysteem een van de eerste systemen is die werden geïmplementeerd, is het inmiddels verouderd. 
De server die verantwoordelijk is voor de communicatie draait op CentOS 6.6 en bied geen ondersteuning meer sinds 30 november 2020 ~\autocite{Reock2020}.
Vervanging door een modern systeem is noodzakelijk, aangezien de veiligheid in het gedrang komt.
Systemen die End of Life (EoL) zijn, ontvangen geen ondersteuning of updates meer van de leverancier, 
wat ze kwetsbaar maakt voor aanvallen.
Hackers kunnen dergelijke servers eenvoudig aanvallen omdat bekende kwetsbaarheden niet langer worden verholpen \autocite{Mittal2024}.
Het vervangen van de server zelf valt buiten de scope van dit onderzoek en zal worden overgedragen aan het IT-infrastructuurteam binnen het bedrijf
\begin{table}[h!]
  \centering
  \begin{tabular}{|c|c|c|}
      \hline
      \textbf{Version} & \textbf{Release Date} & \textbf{End-of-Life Date} \\
      \hline
      CentOS 8 & September 24, 2019 & December 31, 2021 \\
      \hline
      CentOS 7 & July 7, 2014 & June 30, 2024 \\
      \hline
      CentOS 6 & July 10, 2011 & November 30, 2020 \\
      \hline
      CentOS 5 & April 12, 2007 & March 31, 2017 \\
      \hline
  \end{tabular}
  \caption{CentOS Release and End-of-Life Dates~\autocite{Reock2020}}
  \label{tab:centos6}
\end{table}
\newline
 
% Software
De software op die server, SonicMQ 7.0 van Progress Software Corporation, werkt als de ``middleware'' voor de communicatie.
Omdat deze software verouderd is, niet meer ondersteund wordt en niet kan geïnstalleerd worden op een modern OS, moeten we zoeken naar een alternatief.
Dit onderzoek richt zich dan ook op het grondig onderzoeken en vergelijken van geschikte alternatieven die voldoen aan de 
niet-functionele eisen van het automatisch magazijn. Deze vereisten worden later toegelicht in het hoofdstuk methodologie.
\newline

\section{\IfLanguageName{dutch}{Probleemstelling}{Problem Statement}}%
\label{sec:Probleemstelling}

% Uit je probleemstelling moet duidelijk zijn dat je onderzoek een meerwaarde heeft voor een concrete doelgroep. De doelgroep moet goed gedefinieerd en afgelijnd zijn. 
% Doelgroepen als ``bedrijven,'' ``KMO's'', systeembeheerders, enz.~zijn nog te vaag. 
% Als je een lijstje kan maken van de personen/organisaties die een meerwaarde zullen vinden in deze bachelorproef (dit is eigenlijk je steekproefkader), 
% dan is dat een indicatie dat de doelgroep goed gedefinieerd is.
% Dit kan een enkel bedrijf zijn of zelfs één persoon (je co-promotor/opdrachtgever).

% Probleemstelling
Verouderde software is gevoelig voor uitval, kostelijk om te onderhouden en bevat extra complexiteit om te integreren met andere systemen.
Dit beïnvloedt de continuïteit en verhoogt de onderhoudskosten~\autocite{Khadka2016}.
Volgende nadelen van verouderde software doen zich voor: 
\begin{itemize}
  \item De onderhoudskosten, waardoor mensen in de firma specifiek opgeleid moeten worden om deze te onderhouden. 
  \item Beveilgingsrisico's, cybercriminelen richten zich vaak op verouderde software, omdat de kwetsbaarheden publiekelijk bekend zijn en niet meer worden aangepakt.
  \item Betrouwbaarheid, omdat verouderde software een hogere kans heeft op uitval, kan dit leiden tot hogere kosten door verlies van productiviteit en herstelwerkzaamheden.
\end{itemize}

Tijdens een uitval kan het systeem geen data meer versturen, noch ontvangen. 
Hierdoor kunnen bestellingen van klanten vertraging oplopen en kunnen dus de kosten hoog oplopen.
Vervanging van deze software is dus noodzakelijk, waardoor volgende vraag kan gesteld worden: 
\newline
\emph{Welke moderne messaging systeem kan het huidige SonicMQ-systeem tussen de PLC en het WMS vervangen, 
met aandacht voor compatibiliteit, efficiëntie, betrouwbaarheid, veiligheid en kosten?}
\newline

Deze vraag vormt de leidraad voor het onderzoek naar een nieuw systeem die toekomstbestendig is 
en de continuïteit van het automatische magazijn kan waarborgen.

 
\section{\IfLanguageName{dutch}{Onderzoeksvraag}{Research question}}%
\label{sec:Onderzoeksvraag}

% Wees zo concreet mogelijk bij het formuleren van je onderzoeksvraag. 
% Een onderzoeksvraag is trouwens iets waar nog niemand op dit moment een antwoord heeft (voor zover je kan nagaan). 
% Het opzoeken van bestaande informatie (bv. ``welke tools bestaan er voor deze toepassing?'') is dus geen onderzoeksvraag. 
% Je kan de onderzoeksvraag verder specifiëren in deelvragen. 
% Bv.~als je onderzoek gaat over performantiemetingen, dan 

Welke \emph{messaging technologieën} zijn het meest geschikt om het verouderde SonicMQ software te vervangen 
en de continuïteit van automatische magazijn te garanderen?
\newline

Enkele cruciale deelvragen met betrekking tot de hoofdvraag:
\begin{enumerate} 
  \item Wat zijn de non-functional requirements?
  \item Waarom is een weloverwogen keuze van essentieel belang?
  \item Welke messaging systemen zijn er beschikbaar?
  \item Wat is de inspanning om een gekozen technologie te implementeren?
  \item Welk messaging systeem is een weloverwogen keuze?
\end{enumerate}


\section{\IfLanguageName{dutch}{Onderzoeksdoelstelling}{Research objective}}%
\label{sec:Onderzoeksdoelstelling}

% Wat is het beoogde resultaat van je bachelorproef? Wat zijn de criteria voor succes? Beschrijf die zo concreet mogelijk. 
% Gaat het bv.\ om een proof-of-concept, een prototype, een verslag met aanbevelingen, een vergelijkende studie, enz.
% onderzoeksdoelstelling

Het doel van dit onderzoek is om te bepalen welk messaging systeem het meest geschikt is voor de vervanging van de huidige SonicMQ,
die gebruikt wordt voor de communicatie tussen WMS en PLC. 
Hierbij moet er rekening gehouden worden met de niet-functionele vereisten,
zoals flexibiliteit, performantie, beveiliging, integratiemogelijkheden en kosten.
Dit onderzoek zal resulteren in een rapport met aanbevelingen.

\section{\IfLanguageName{dutch}{Opzet van deze bachelorproef}{Structure of this bachelor thesis}}%
\label{sec:Opzet-bachelorproef}

% Het is gebruikelijk aan het einde van de inleiding een overzicht te
% geven van de opbouw van de rest van de tekst. Deze sectie bevat al een aanzet
% die je kan aanvullen/aanpassen in functie van je eigen tekst.

De rest van deze bachelorproef is als volgt opgebouwd:
\begin{enumerate}
  \item In Hoofdstuk~\ref{ch:stand-van-zaken} wordt een overzicht gegeven van de stand van zaken binnen het onderzoeksdomein, op basis van een literatuurstudie.
  \item In Hoofdstuk~\ref{ch:methodologie} wordt de methodologie toegelicht en worden de gebruikte onderzoekstechnieken besproken om een antwoord te kunnen formuleren op de onderzoeksvragen.
  % TODO: Vul hier aan voor je eigen hoofstukken, één of twee zinnen per hoofdstuk
  \begin{enumerate}
    \item Vul hier aan voor je eigen hoofstukken, één of twee zinnen per hoofdstuk
    \item \dots
  \end{enumerate}
  \item In Hoofdstuk~\ref{ch:conclusie}, tenslotte, wordt de conclusie gegeven en een antwoord geformuleerd op de onderzoeksvragen. Daarbij wordt ook een aanzet gegeven voor toekomstig onderzoek binnen dit domein.
\end{enumerate}
